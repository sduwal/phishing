%%%%%%%%%%%%%%%%%%%%%%%%%%%%%%%%%%%%%%%%%%%%%%%%%%%
%
%  New template code for TAMU Theses and Dissertations starting Spring 2021.  
%
%
%  Last Updated: 1/13/2021
%
%%%%%%%%%%%%%%%%%%%%%%%%%%%%%%%%%%%%%%%%%%%%%%%%%%%

% THIS TEMPLATE IS THE MOST
% CURRENT. SEE THE FILES README.TXT AND NEWCHANGES.TXT
% FOR MORE INFORMATION.

\documentclass[12pt]{report}

\usepackage{tamuconfig}
\usepackage{array}
\usepackage{booktabs}
\usepackage{listings}
\usepackage{xcolor}
\usepackage{subfig}
\definecolor{codegreen}{rgb}{0,0.6,0}
\definecolor{codegray}{rgb}{0.5,0.5,0.5}
\definecolor{codepurple}{rgb}{0.58,0,0.82}
\definecolor{backcolour}{rgb}{0.95,0.95,0.92}
\definecolor{lightgray}{rgb}{.9,.9,.9}
\definecolor{darkgray}{rgb}{.4,.4,.4}
\definecolor{purple}{rgb}{0.65, 0.12, 0.82}
\usepackage[bottom]{footmisc}

\lstdefinestyle{Python}{
    backgroundcolor=\color{backcolour},   
    commentstyle=\color{codegreen},
    keywordstyle=\color{magenta},
    numberstyle=\tiny\color{codegray},
    stringstyle=\color{codepurple},
    basicstyle=\ttfamily\footnotesize,
    breakatwhitespace=false,         
    breaklines=true,                 
    captionpos=b,                    
    keepspaces=true,                 
    numbers=left,                    
    numbersep=5pt,                  
    showspaces=false,                
    showstringspaces=false,
    showtabs=false,                  
    tabsize=2
}
\lstset{style=Python}


\lstdefinelanguage{JavaScript}{
  keywords={typeof, new, true, false, catch, function, return, null, catch, switch, var, if, in, while, do, else, case, break},
  keywordstyle=\color{blue}\bfseries,
  ndkeywords={class, export, boolean, throw, implements, import, this},
  ndkeywordstyle=\color{darkgray}\bfseries,
  identifierstyle=\color{black},
  sensitive=false,
  comment=[l]{//},
  morecomment=[s]{/*}{*/},
  commentstyle=\color{purple}\ttfamily,
  stringstyle=\color{red}\ttfamily,
  morestring=[b]',
  morestring=[b]"
}

\lstset{
   language=JavaScript,
   backgroundcolor=\color{lightgray},
   extendedchars=true,
   basicstyle=\tiny,
   showstringspaces=false,
   showspaces=false,
   numbers=left,
   numberstyle=\footnotesize,
   numbersep=5pt,
   tabsize=2,
   breaklines=true,
   showtabs=false,
   captionpos=b
}

\newcolumntype{L}{>{\centering\arraybackslash}m{3cm}}

% Most of the packages that set the default settings
% for the document have moved to the style file
% tamuconfig.sty. This includes

%These next lines change the font. Fixes for certain
%fonts will be implemented in a future release.

%Comment this line if you do not wish to use Times
%New Roman. The font used will then be the LaTeX
%default of Computer Modern.
\usepackage{times}
%\usepackage{cmbright}
\usepackage[T1]{fontenc}

% For natbib-style references, uncomment this.
%\usepackage{natbib}

%This package allows for the use of graphics in the
%document.
\usepackage{graphicx}

%If you have JPEG format images, add .jpg as an
%allowed file extension below. Same for Bitmaps (.bmp).
\DeclareGraphicsExtensions{.png}

%It is best practice to keep all your pictures in
%one folder inside the main directory in which your
%TeX file is kept. Here the folder is named "graphic."
%Replace the name here with your folder's name, if needed.
%The period is needed due to relative referencing.
\graphicspath{ {./figures/} }

% For quick document navigation.
\usepackage[hidelinks]{hyperref}
\usepackage{nameref}
\newcounter{mylabelcounter}

\makeatletter
\newcommand{\labelText}[2]{%
#1\refstepcounter{mylabelcounter}%
\immediate\write\@auxout{%
  \string\newlabel{#2}{{1}{\thepage}{{\unexpanded{#1}}}{mylabelcounter.\number\value{mylabelcounter}}{}}%
}%
}
\makeatother


%%%%%%%%%%%%%%%%%%%%%%%%%%%%%%%%%%%%%%%%%%%%%%%%%%%%%%%%%
%Please place all your personal packages here. Check to
%see if the packages you wish to use are not already
%declared above. Placing all your personal packages
%here allows me to determine if there are any package
%issues in compilation, as well as any conflicts
%that may arise by the order of loading.
%
%%%%%%%%%%%%%%%%%%%%%%%%%%%%%%%%%%%%%%%%%%%%%%%%%%%%%%%%%
%%%%%%%%%%%%%%%%%%%%%%%%%%%%%%%%%%%%%%%%%%%%%%%%%%%%%%%%%
%Begin student defined packages.
%%%%%%%%%%%%%%%%%%%%%%%%%%%%%%%%%%%%%%%%%%%%%%%%%%%%%%%%%
%%%%%%%%%%%%%%%%%%%%%%%%%%%%%%%%%%%%%%%%%%%%%%%%%%%%%%%%%
%End student defined packages.
%%%%%%%%%%%%%%%%%%%%%%%%%%%%%%%%%%%%%%%%%%%%%%%%%%%%%%%%%

% End preamble. Document begins below.

\begin{document}
\providecommand{\tabularnewline}{\\}
\begin{titlepage}
\begin{center}\large
Thesis Title
\vspace{5em}

A Thesis

\vspace{5em}

\begin{singlespace}

Submitted to the Graduate Faculty of the  \\
University of New Orleans \\
in partial fulfillment of the \\
requirements for the degree of
\end{singlespace}

\vspace{5em}
\begin{singlespace}
Master of Science \\
in \\
Computer Science
\end{singlespace}

\vspace{5em}
\par\end{center}
\begin{center}\large
\hspace{12pt} by \\
\hspace{12pt} Saroj Duwal \\


\par\end{center}
\end{titlepage}
\pagebreak{}




 % This is simply a file that formats and adds your titlepage, please do not edit this unless you have a specific need. .

% \include{data/acknowledgements}

\include{data/lists}  % This is simply a file that formats and adds your toc, lof, and lot, please do not edit this unless you have a specific need.
% 
%%%%%%%%%%%%%%%%%%%%%%%%%%%%%%%%%%%%%%%%%%%%%%%%%%%
%%%%%%%%%%%%%%%%%%%%%%%%%%%%%%%%%%%%%%%%%%%%%%%%%%%%%%%%%%%%%%%%%%%%%
%%                           ABSTRACT
%%%%%%%%%%%%%%%%%%%%%%%%%%%%%%%%%%%%%%%%%%%%%%%%%%%%%%%%%%%%%%%%%%%%%

\chapter*{Abstract}
\addcontentsline{toc}{chapter}{Abstract} % Needs to be set to part, so the TOC doesn't add 'CHAPTER ' prefix in the TOC.

\pagestyle{plain} % No headers, just page numbers
\pagenumbering{roman} % Roman numerals
\setcounter{page}{8}
This is place holder for the abstact sectionLaboris sit sunt nulla elit nisi adipisicing incididunt esse. Occaecat anim ex fugiat aute fugiat enim cillum. Magna do pariatur voluptate cupidatat consequat magna enim occaecat tempor laborum ea aliqua amet. Ex tempor nulla est amet irure dolor veniam occaecat do deserunt cillum ullamco occaecat.



\textbf{Keywords}: keywords, keywords
\pagebreak{}

%%%%%%%%%%%%%%%%%%%%%%%%%%%%%%%%%%%%%%%%%%%%%%%%%%%%%%%%%%%%%%%%%%%%%%
%%                           INTRODUCTION
%%%%%%%%%%%%%%%%%%%%%%%%%%%%%%%%%%%%%%%%%%%%%%%%%%%%%%%%%%%%%%%%%%%%%


\pagestyle{plain} % No headers, just page numbers
\pagenumbering{arabic} % Arabic numerals
\setcounter{page}{1}


\chapter{Introduction}
The rapid internet adoption in everyday life and the workplace has presented us with new security challenges. Users are more active on the internet, giving attackers more opportunities to attack unsuspecting victims. There are various technical security measures such as firewall, encryption, threat hunting software, and engaging automation to mitigate these challenges. However, studies have shown that the human layer is the weakest link in the security chain \cite{jampen} and attackers usually start by targeting the most vulnerable link before performing other detrimental attacks. These attacks with human interaction are generally known as "Social Engineering Attacks." Prevalent social engineering attacks such as phishing, pretexting, baiting, quid pro quo, and tailgating use psychological manipulation to trick users into making security mistakes or giving away sensitive information. This thesis will focus on phishing and different detection techniques through our role-playing gameplay.

\section{What is phishing?}
Phishing is one of the most prevalent social engineering attacks in which attackers target users by contacting them through email, telephone, or text message by attackers posing as a legitimate entity \cite{phishing, apwg}. Unfortunately, these attacks are challenging to detect as attackers use the computing infrastructure to fool the victim into doing something but are doing something else while the computing system is working as intended. Due to this, even users with a high-end security system can be victims. An example of such is the infamous case of John Podesta \cite{Podesta}, Hilary Clinton's campaign chairman for the 2016 presidential election. The "googlemail.com" in the domain successfully tricked John Podesta and the Clinton campaign's computer help desk to trust the email (See fig:\ref{fig:Podesta}).

Phishing attacks are constantly evolving with different tricks. For example, although Podesta's email shows it was initially generated from "googlemail.com," making it seem like it might be from Google, that might not be true. Attackers can use different spoofing techniques to hide the sender's identity. Another common trick attackers use (also present in Podesta's email) is to confuse the user with links hidden behind some text/button or confuse the user with redirecting links (example: TinyURL). As a result, the displayed text/link might not be the final destination. Podesta's team's failure to deal with this phishing email led to leaks of more than 11,000 emails which included private conversations with 2016 presidential nominee Hillary Clinton \cite{anderson_2016}.

\begin{figure}[ht]
    \centering
    \includegraphics[scale=0.7]{{./section1/podesta.png}}
    \caption[Phishing email sent to John Podesta]{Phishing email sent to John Podesta}
    \label{fig:Podesta}
\end{figure}

Successful phishing attacks are very costly to organizations. In 2020, phishing attacks cost US businesses more than \$1.8 billion, up from \$1.7 billion in 2019 \cite{vade}. These attacks can lead to credential/account compromise, giving the attacker access to sensitive information. Attackers may try to use these data for extortion. For example: In 2014, an attack was successful on an invasion of celebrity iCloud accounts, leading to the embarrassing leaking of nude photos. The leak was initially considered due to a breach on Apple services, but it was later a phishing attack pretending to be Apple and Google and asking them to change their password \cite{duke_2014, guardian_2014}.

Phishing attacks are continuously rising and have doubled since early 2020. In July 2021 alone, APWG saw 260,642 phishing attacks \cite{apwg}. Additionally, Proofpoint found that more than 75\% of organizations faced phishing attacks in 2021 \cite{proofpoint}. These uprising trends in attacks have shown some serious need for mitigations for phishing attacks.

\section{Current Mitigations}
The prevention of phishing attacks can be divided into three steps \cite{vayansky}. The first step to stop a phishing attack is preventing the attack from reaching the end-user. We have seen multiple studies on phishing prevention with the help of the machine learning models \cite{yang_zheng_wu_wu_wang_2021, sahingoz_buber_demir_diri_2019}. Machine learning approaches such as K-nearest, XGBoost, CNN, RCNN, Random forest, etc., are commonly used to detect patterns and generalize phishing attacks. Some of the models have shown promises with more than 90\% accuracy—however, a study conducted by What.Hack has shown that only one of the ten anti-phishing tools tested was able to identify over 90\% of phishing websites correctly, and that tool also incorrectly identified 42\% of legitimate websites as fraudulent \cite{what_hack}. Moreover, attackers are always looking for the best way to bypass these automated systems and develop new techniques if automated systems start flagging their attacks. The evolving nature of phishing attacks calls for an additional layer of security on top of the prevention layer.

If the attacks reach the user, the next step to secure the user is by warning the user. Most modern web browsers and email clients warn users of any suspicious activities they detect. For example, the browser will actively warn users with pop up for probable phishing sites. In addition, browsers provide passive hints to understand links better. Browsers use different shades of white to inform the user about a "fully qualified domain name (FQDN)" (also called absolute domain name), the complete domain name for a specific host on the internet. Figure \ref*{fig:browser_cues} shows a use case for such a hint. Attackers will intentionally have a confusing link to trick users into clicking the link. For example, although "help.google.com.bubble.com/changepassword" seems like an email from Google, the actual domain is bubble.com. Users can add any subdomain to domains they own, such as help.google.com.bubble.com, which can potentially be used in phishing attacks.Modern email clients provide similar hints for spam emails and notify the users if they can not verify the sender. Active warnings are more effective than passive signs \cite{vayansky}, but attackers can easily bypass these warnings by creating new sites and context-aware websites or emails every time they are flagged.

\begin{figure}[h]
    \centering
    \includegraphics[scale=0.7]{{./section1/browser_cues.png}}
    \caption[Browser cues on links]{Browsers uses different shades to indicate the primary link.}
    \label{fig:browser_cues}
\end{figure}

The final step to avoid phishing emails is user training. A study done by Proofpoint shows that 34\% of US respondents believe emails with familiar logos are safe \cite{proofpoint}. The study indicates a general lack of awareness about phishing campaigns among the general population. There are many tools used for phishing training. One of the most common tools to train users is cyber security videos and reading materials. However,  Kumaraguru et al. saw that users seldom seek these materials and tend to ignore emails directing them to these materials \cite{johnny_phishing}. In addition, they noticed that most users do not spend much time reading security-related tutorials. This calls for an interactive training program to keep the user focused and engaged during training.

We have seen new and existing training materials incorporating gaming techniques. Gamification has been gaining rapid popularity over the past decade\cite{schultz_2021}. It increases engagement by incentivizing learners to pay attention and complete activities. We can observe existing training videos incorporating gaming techniques, such as letting users choose the correct option in the middle of training videos (a mini quiz game) and giving badges after completion. Newer training videos take gamification further and let learners play through various scenarios, make choices and see the rewards or consequences of their decision. For example, Infosec's Choose Your Own Adventure Security Awareness Game\cite{infosec_2022} has interactive storytelling to keep the user focused till the end of the video.

Gamification has improved the interactivity with the user, but existing training videos fail to cover the technical details that are commonly found in phishing emails. Our gameplay covers various technical aspects commonly found in phishing emails, such as domains, spoofing, and link hiding techniques attackers use to trick users.

\section{Litearature Review}
\subsection{Serious games}
The gaming approach in education is not novel and has been used for over a decade \cite{almeida_2012}. There is a dedicated genre of games (typically online applications), termed as serious games, dedicated to using video games to communicate specific information that helps introduce relevant concepts and apply those concepts to solve problems.  The primary purpose of these games is not entertainment but to promote learning. Because of game design techniques, users are more engaged and immersed in playing by using rewards, story progression, or other feedback systems. Virtual world also provides users with a safe space to experiment without real-life consequences.

We can see serious games in many fields such as education, healthcare, training, and consultancy. Below are examples of some serious games targeted at different audiences:
\begin{itemize}
    \item “Garfield’s Count Me In”\cite{count_me_in} helps children in (special education) primary school practice their arithmetic skills.  This math game contains different exercises or ‘brick’, which form the foundation for a new layer of exercises. The game design help students master the first layer of exercises before moving to the next layer.

    \item “Killer Flu”\cite{killer_flu} (one of many games by “Persuasive Games”) is an attempt to explain how flu mutates and spreads and how challenging it can be for a deadly strain to affect a large population geographically. The player takes the role of the flu itself, trying to mutate and then spread it in a variety of conditions.

    \item  “Pulse!!”\cite{pulse} is a serious training game that simulates nearly every aspect of modern surgery. The game gives the player a chance to move through case-based learning modules and practice on a virtual patient. This gives the player to hone their skills without harming real patients.
\end{itemize}

\begin{figure}[h]
    \centering
    \begin{minipage}{0.32\textwidth}
        \centering
        \includegraphics[width=0.9\textwidth]{./section1/garfield.jpeg}
    \end{minipage} \hfill
    \begin{minipage}{0.32\textwidth}
        \centering
        \includegraphics[width=0.9\textwidth]{./section1/killer_flu.jpg} % second figure itself
    \end{minipage} \hfill
    \begin{minipage}{0.32\textwidth}
        \centering
        \includegraphics[width=0.9\textwidth]{./section1/pulse.jpg} % second figure itself
    \end{minipage}
    \caption[Example of serious games]{Screenshot of serious games. From left to right: a. Garfield’s Count Me, b. Killer Flu, c. Pulse!!}
\end{figure}

There have been various attempts to use serious games as a practical phishing training module. Hendrix et al. \cite{hendrix_al_sherbaz_bloom_2016} compared the effectiveness of cyber security training tools with some popular games designed for cyber security training and found some positive signs.

We will discuss some previous work below, separated by category.

\subsection{Board and card games}
There have been some studies based on non-computer-based games. For example, Control-Alt-Hack \cite{control_alt_hack} tries to educate the user with the help of a card game. Similarly, "Smells Phishy?"\cite{smels_phishy} is another non-computer-based game that tries to raise awareness about online phishing scams. Both these game depends on cards to divide the task and learn skills. Although both the games had shown promise in their approach, non-computer-based games have some inherent limitations. The games require pre-setup (with the need for the cards and boards) and make it harder to use than computer-based games. Furthermore, the current approaches only teach users about phishing attacks. The limited skills these games provide may not be best suited as an individual training module.

\subsection{Phishing Link training}
There have been numerous computer games about phishing. One common category many studies focus on is training users to verify phishing links. Anti-Phishing Phil \cite{anti_phishing_phil} is one of the pioneers in this field. Their gameplay puts the user as a fish. The goal of the fish is to grow larger by eating the good bugs (non-phishing links). Phish Phinder \cite{phish_phinder} is another example of link based training game that has similar gameplay and story to Anti-Phishing Phil but builds upon the game with more levels and interaction. Baral et al. \cite{gamified_appraoch} has a similar concept with a balloon shooting game. The goal of these games is to differentiate the phishing links and actual links. However, these games do not consider the context of the email. Attackers use psychological manipulation such as creating a sense of urgency, fake giveaways or making it seem like an email from somebody you know to trick people into clicking these links.

\subsection{Role playing game}
"What.Hack" \cite{what_hack} saw the shortcomings of the link-based game and developed gameplay that train the user on links as well as email context. "What.Hack" puts the user as a player required to process emails to acquire contracts and protect their network from cybercriminals. It approaches the training by having the user role play as a victim and look at different techniques that could be found in actual attacks.

The contextual emails addressed one of the most significant shortcomings of link-based games. The game successfully educated users with similar concepts as link-based games and added context to the links. The result from "What.Hack" clearly shows users' preference for their gameplay compared to Anti-Phishing Phil. Moreover, the game demonstrated significant improvement compared to other games in detecting phishing emails \cite{what_hack}.

\section{Objective}
"What.Hack" clearly demonstrated that role-playing games with contextual emails were more effective than existing gameplays. Unfortunately, we could not find any other significant study that tried to build upon this finding. Therefore, we have developed gameplay inspired by "What.Hack" but approached the role-playing aspect as an attacker instead of a victim.

General phishing training such as videos and reading materials has taught users what to look for as victims. However, we believe looking at the attacker's perspective will help users understand what the attacker might look for while creating a phishing email. This will also help complement the currently available training games.

Our goals for the study can be summarized as:

\begin{itemize}
    \setlength\itemsep{-0.6em}
    \item Develop a role-playing game to train the users about phishing through an attackers perspective
    \item Compare our results with the existing study
\end{itemize}

% %%%%%%%%%%%%%%%%%%%%%%%%%%%%%%%%%%%%%%%%%%%%%%%%%%%%%%%%%%%%%%%%%%%%%%%
%%%                           System Description
%%%%%%%%%%%%%%%%%%%%%%%%%%%%%%%%%%%%%%%%%%%%%%%%%%%%%%%%%%%%%%%%%%%%%%




\chapter{ System Description} \label{cha:SystemDescription}
This is a place holder for system design
% %%%%%%%%%%%%%%%%%%%%%%%%%%%%%%%%%%%%%%%%%%%%%%%%%%%%%%%%%%%%%%%%%%%%%%%
%%%                           System Description
%%%%%%%%%%%%%%%%%%%%%%%%%%%%%%%%%%%%%%%%%%%%%%%%%%%%%%%%%%%%%%%%%%%%%%




\chapter{Evaluation}

\section{Insights}
- Subdomains and confusion about Subdomains
- spoofing and confusion

% %%%%%%%%%%%%%%%%%%%%%%%%%%%%%%%%%%%%%%%%%%%%%%%%%%%%%%%%%%%%%%%%%%%%%%%
%%%                           System Description
%%%%%%%%%%%%%%%%%%%%%%%%%%%%%%%%%%%%%%%%%%%%%%%%%%%%%%%%%%%%%%%%%%%%%%




\chapter{Discussion}
\section{Limitations}
\section{Future Work}
\section{Conclusion}
% \include{data/section5}
%The next line is the format for inserting new sections.
%Replace the name "newsection"  with the name of your
%new section file.
%\include{data/newsection}

%fix spacing in bibliography, if any...
%%%%%%%%%%%%%%%%%%%%%%%%%%%%%%%%%%%%%%%%%%%%%%%%%%%%%%%%%%%%%
\let\oldbibitem\bibitem
\renewcommand{\bibitem}{\setlength{\itemsep}{0pt}\oldbibitem}
%%%%%%%%%%%%%%%%%%%%%%%%%%%%%%%%%%%%%%%%%%%%%%%%%%%%%%%%%%%%%%%
%The bibliography style declared is the IEEE format. If
%you require a different style, see the document
%bibstyles.pdf included in this package. This file,
%hosted by the University of Vienna, shows several
%bibliography styles and examples of in-text citation
%and a references page.
\bibliographystyle{ieeetr}

\phantomsection
\addcontentsline{toc}{chapter}{References}

\renewcommand{\bibname}{{\Large\bf References}}

%This file is a .bib database that contains the sources.
%This removes the dependency on the previous file
%bibliography.tex.
\bibliography{data/myReference}




%This next line includes appendices. The file
%appendix.tex contains commands pointing to
%the appendix files; be sure to change these
%pointers if you end up changing the filenames.
%Leave this commented if you will not need
%appendix material.
% %%%%%%%%%%%%%%%%%%%%%%%%%%%%%%%%%%%%%%%%%%%%%%%%%%%
%
%  New template code for TAMU Theses and Dissertations starting Spring 2021.  
%
%
%  Author: Thesis Office
%  
%  Last Updated: 1/13/2021
%
%%%%%%%%%%%%%%%%%%%%%%%%%%%%%%%%%%%%%%%%%%%%%%%%%%%

\begin{appendices}
\titleformat{\chapter}{\centering\Large}{Appendix \thechapter}{0em}{\vskip .5\baselineskip\centering}
\renewcommand{\appendixname}{Appendix}

%%%%%%%%%%%%%%%%%%%%%%%%%%%%%%%%%%%%%%%%%%%%%%%%%%%%%%%%%%%%%%%%%%%%%%
%%                           APPENDIX A 
%%%%%%%%%%%%%%%%%%%%%%%%%%%%%%%%%%%%%%%%%%%%%%%%%%%%%%%%%%%%%%%%%%%%%

\phantomsection

\chapter{Code snippets}
\label{chap:codesnippets}

\begin{lstlisting}[language=JavaScript, caption=DynamicCompressor fingerprint generation code, label=dynamicCompressorFingerprintGenerationCode]
  getDynamicCompressorFingerprint(): Promise<any>  {
    let sumBuffer = 0;
    let sumBufferHash = null;
    return new Promise((resolve, reject) => {
      try {
        let offlineAudioCtx = new ((<any>window).OfflineAudioContext || (<any>window).webkitOfflineAudioContext)(OFFLINEAUDIOCTX.numberOfChannels, OFFLINEAUDIOCTX.length, OFFLINEAUDIOCTX.sampleRate);
        if (offlineAudioCtx) {
          let oscillator =  offlineAudioCtx.createOscillator();
          oscillator.type = SIGNAL_TYPE;
          oscillator.frequency.value = FREQUENCY;
          
          // Create and configure compressor
          let compressor = offlineAudioCtx.createDynamicsCompressor();
          compressor.threshold && (compressor.threshold.value = COMPRESSOR.threshold);
          compressor.knee && (compressor.knee.value = COMPRESSOR.knee);
          compressor.ratio && (compressor.ratio.value = COMPRESSOR.ratio);
          // compressor.reduction && (compressor.reduction.value = -20);
          compressor.attack && (compressor.attack.value = COMPRESSOR.attack);
          compressor.release && (compressor.release.value = COMPRESSOR.release);

          // Connect nodes
          oscillator.connect(compressor);
          compressor.connect(offlineAudioCtx.destination);

          // Start audio processing
          oscillator.start(0);
          offlineAudioCtx.startRendering();
          offlineAudioCtx.oncomplete = ((evnt: any) => {
            sumBuffer = 0;
            let MD5 =  CryptoJS.algo.MD5.create();
            for (let i = 0; i < evnt.renderedBuffer.length; i++) {
              MD5.update(evnt.renderedBuffer.getChannelData(0)[i].toString());
            }
            const hash = MD5.finalize();
            sumBufferHash = hash.toString(CryptoJS.enc.Hex);
            for (let i = 4500; 5e3 > i; i++) {
              sumBuffer += Math.abs(evnt.renderedBuffer.getChannelData(0)[i]);
            }
            oscillator.disconnect();
            //console.log({"dynamicCompressor": sumBufferHash, "sum": sumBuffer});
            //alert("dynamiccompressor " +sumBufferHash)
            resolve({"hash": sumBufferHash, "sum": sumBuffer, "noFingerprint": false});
          });
        } else {
          reject({"hash": sumBufferHash, "sum": sumBuffer, "noFingerprint": true});
        }
      } catch (u) {
        reject({"hash": sumBufferHash, "sum": sumBuffer, "noFingerprint": true});
      }
    });
  }
\end{lstlisting}

\begin{lstlisting}[language=JavaScript, caption=OscillatorNode fingerprint generation code, label=oscillatorNodeFingerprintGenerationCode]
  getOscillatorNodeFingerprint(): Promise<any>  {
    let oscillatorNode = [];
    let hash = null;
    return new Promise((resolve, reject) => {
      try {
        let audioCtx = new ((<any>window).AudioContext || (<any>window).webkitAudioContext)();
        if (audioCtx) {
          let oscillator = audioCtx.createOscillator();
          let analyser = audioCtx.createAnalyser();
          let gain = audioCtx.createGain();
          let scriptProcessor = audioCtx.createScriptProcessor(SCRIPT_PROCESSOR.bufferSize, SCRIPT_PROCESSOR.numberOfInputChannels, SCRIPT_PROCESSOR.numberOfOutputChannels);
          gain.gain.value = 1; // Disable volume
          analyser.fftSize = 2048;
          oscillator.type = SIGNAL_TYPE // Set oscillator to output wave
          oscillator.connect(analyser); // Connect oscillator output to analyser input
          analyser.connect(scriptProcessor); // Connect analyser output to scriptProcessor input
          scriptProcessor.connect(gain); // Connect scriptProcessor output to gain input
          gain.connect(audioCtx.destination); // Connect gain output to audiocontext destination
          scriptProcessor.onaudioprocess = (async (event) => {
              const bins = new Float32Array(analyser.frequencyBinCount);
              analyser.getFloatFrequencyData(bins);
              for (let i = 0; i < bins.length; i++) {
                oscillatorNode.push(bins[i]);
              }
              analyser.disconnect();
              scriptProcessor.disconnect();
              gain.disconnect();
              const audioFP = JSON.stringify(oscillatorNode);
              hash = CryptoJS.MD5(audioFP).toString();
              await audioCtx.close();
              resolve({"hash": hash, "values": oscillatorNode, "noFingerprint": false});
          });
          oscillator.start(0);
        } else {
          reject({"hash": hash, "values": oscillatorNode, "noFingerprint": true});
        }
      } catch (u) {
        reject({"hash": hash, "values": oscillatorNode, "noFingerprint": true});
      }
    });
  }
\end{lstlisting}

\begin{lstlisting}[language=JavaScript, caption=Hybrid fingerprint generation code, label=hybridFingerprintGenerationCode]
  getHybridFingerprintWithAudioCtx(): Promise<any>  {
    let hybridOscillatorNode = [];
    let hybridHash = null;
    return new Promise((resolve,reject) => {
      try {
        let audioCtx = new ((<any>window).AudioContext || (<any>window).webkitAudioContext)();
        if (audioCtx) {
          let oscillator = audioCtx.createOscillator();
          let analyser = audioCtx.createAnalyser();
          let gain = audioCtx.createGain();
          let scriptProcessor = audioCtx.createScriptProcessor(SCRIPT_PROCESSOR.bufferSize, SCRIPT_PROCESSOR.numberOfInputChannels, SCRIPT_PROCESSOR.numberOfOutputChannels);

          // Create and configure compressor
          let compressor = audioCtx.createDynamicsCompressor();
          compressor.threshold.setValueAtTime(COMPRESSOR.threshold, audioCtx.currentTime);
          compressor.knee.setValueAtTime(COMPRESSOR.knee, audioCtx.currentTime);
          compressor.ratio.setValueAtTime(COMPRESSOR.ratio, audioCtx.currentTime);
          compressor.attack.setValueAtTime(COMPRESSOR.attack, audioCtx.currentTime);
          compressor.release.setValueAtTime(COMPRESSOR.release, audioCtx.currentTime);

          gain.gain.value = 0; // Disable volume
          analyser.fftSize = 2048;
          oscillator.type = SIGNAL_TYPE // Set oscillator to output triangle wave
          oscillator.connect(compressor); // Connect oscillator output to dynamic compressor
          compressor.connect(analyser); // Connect compressor to analyser
          analyser.connect(scriptProcessor); // Connect analyser output to scriptProcessor input
          scriptProcessor.connect(gain); // Connect scriptProcessor output to gain input
          gain.connect(audioCtx.destination); // Connect gain output to audiocontext destination
          scriptProcessor.onaudioprocess = (async (bins: any) => {
              bins = new Float32Array(analyser.frequencyBinCount);
              analyser.getFloatFrequencyData(bins);
              for (let i = 0; i < bins.length; i++) {
                  hybridOscillatorNode.push(bins[i]);
              }
              analyser.disconnect();
              scriptProcessor.disconnect();
              gain.disconnect();
              const audioFP = JSON.stringify(hybridOscillatorNode);
              hybridHash = CryptoJS.MD5(audioFP).toString();
              await audioCtx.close();
              resolve({"hash": hybridHash, "values": hybridOscillatorNode, "noFingerprint": false});
          });
          oscillator.start(0);
        } else {
          reject({"hash": hybridHash, "values": hybridOscillatorNode, "noFingerprint": true});
        }
      } catch (u) {
        reject({"hash": hybridHash, "values": hybridOscillatorNode, "noFingerprint": true});
      }
    });
  }
\end{lstlisting}

\begin{lstlisting}[language=JavaScript, caption=Custom Signal Hybrid fingerprint generation code, label=customSignalHybridFingerprintGenerationCode]
  getCustomSignalHybridFingerprintAudioCtx(): Promise<any>  {
    let hybridOscillatorNode = [];
    let hybridHash = null;
    return new Promise(async (resolve, reject) => {
      try {
        let audioCtx = new ((<any>window).AudioContext || (<any>window).webkitAudioContext)();
        if (audioCtx) {
          const OFFSET = 0.7;
          const pi = Math.PI;
          // https://medium.com/web-audio/phase-offsets-with-web-audio-wavetables-c7dc85ac3218
          // https://meettechniek.info/additional/additive-synthesis.html
          const real = new Float32Array(11);
          const imag = new Float32Array(11);
          real[0] = 0.360;
          real[1] = 0.760;
          real[2] = 0.120;
          real[3] = 0.745;
          real[4] = 0.235;
          real[5] = 0.145;
          real[6] = 0.545;
          real[7] = 0.675;
          real[8] = 0.585;
          real[9] = 0.685;
          real[10] = 0.115;
          real[11] = 0.660;

          imag[0] = pi/2;
          imag[1] = 0;
          imag[2] = pi/2;
          imag[3] = 0;
          imag[4] = pi/2;
          imag[5] = 0;
          imag[6] = pi/2;
          imag[7] = 0;
          imag[8] = pi/2;
          imag[9] = 0;
          imag[10] = pi/2;
          imag[11] = 0

          const wave = audioCtx.createPeriodicWave(real, imag, { disableNormalization: true });
          let oscillator =  audioCtx.createOscillator();
          oscillator.frequency.value = 440;
          oscillator.setPeriodicWave(wave);
          const offset = audioCtx.createConstantSource();
          offset.offset.value = OFFSET;
          let analyser = audioCtx.createAnalyser();
          let gain = audioCtx.createGain();
          let scriptProcessor = audioCtx.createScriptProcessor(SCRIPT_PROCESSOR.bufferSize, SCRIPT_PROCESSOR.numberOfInputChannels, SCRIPT_PROCESSOR.numberOfOutputChannels);

          let compressor = audioCtx.createDynamicsCompressor();
          compressor.threshold.setValueAtTime(COMPRESSOR.threshold, audioCtx.currentTime);
          compressor.knee.setValueAtTime(COMPRESSOR.knee, audioCtx.currentTime);
          compressor.ratio.setValueAtTime(COMPRESSOR.ratio, audioCtx.currentTime);
          compressor.attack.setValueAtTime(COMPRESSOR.attack, audioCtx.currentTime);
          compressor.release.setValueAtTime(COMPRESSOR.release, audioCtx.currentTime);

          gain.gain.value = 0; // Disable volume
          analyser.fftSize = 2048;
          oscillator.connect(compressor); // Connect oscillator output to dynamic compressor
          compressor.connect(analyser); // Connect oscillator output to dynamic compressor
          analyser.connect(scriptProcessor); // Connect analyser output to scriptProcessor input
          scriptProcessor.connect(gain); // Connect scriptProcessor output to gain input
          gain.connect(audioCtx.destination); // Connect gain output to audiocontext destination

          scriptProcessor.onaudioprocess = (async (bins: any) => {
            bins = new Float32Array(analyser.frequencyBinCount);
            analyser.getFloatFrequencyData(bins);
            for (let i = 0; i < bins.length; i++) {
              hybridOscillatorNode.push(bins[i]);
            }
            analyser.disconnect();
            scriptProcessor.disconnect();
            gain.disconnect();
            const audioFP = JSON.stringify(hybridOscillatorNode);
            hybridHash = CryptoJS.MD5(audioFP).toString();
            await audioCtx.close();
            resolve({"hash": hybridHash, "values": hybridOscillatorNode, "noFingerprint": false});
          });
          oscillator.start(0);
          offset.start();
        } else {
          reject({"hash": hybridHash, "values": hybridOscillatorNode, "noFingerprint": true});
        } 
      } catch (u) {
        reject({"hash": hybridHash, "values": hybridOscillatorNode, "noFingerprint": true});
      }
    });
  }
\end{lstlisting}

\begin{lstlisting}[language=JavaScript, caption=Audio Source Hybrid fingerprint generation code, label=audioSourceHybridFingerprintGenerationCode]
  getAudioSourceHybridFingeprintAudioCtx(): Promise<any> {
    let audioData = null;
    let hash= null;
    let analyserNodeData = [];
    return new Promise((resolve, reject) => {
      try {
        const audioContext = new ((<any>window).AudioContext || (<any>window).webkitAudioContext)();
        if (audioContext) {
          const sourceNode = audioContext.createBufferSource();
          const analyserNode = audioContext.createAnalyser();
          const gain = audioContext.createGain();
          const scriptProcessor = audioContext.createScriptProcessor(SCRIPT_PROCESSOR.bufferSize, SCRIPT_PROCESSOR.numberOfInputChannels, SCRIPT_PROCESSOR.numberOfOutputChannels);
           // Create and configure compressor
          const compressor = audioContext.createDynamicsCompressor();
          compressor.threshold.setValueAtTime(COMPRESSOR.threshold, audioContext.currentTime);
          compressor.knee.setValueAtTime(COMPRESSOR.knee, audioContext.currentTime);
          compressor.ratio.setValueAtTime(COMPRESSOR.ratio, audioContext.currentTime);
          compressor.attack.setValueAtTime(COMPRESSOR.attack, audioContext.currentTime);
          compressor.release.setValueAtTime(COMPRESSOR.release, audioContext.currentTime);
          gain.gain.value = 0; // Disable volume
          analyserNode.fftSize = 2048;

          // Now connect the nodes together
          sourceNode.connect(compressor);
          compressor.connect(analyserNode);
          analyserNode.connect(scriptProcessor);
          scriptProcessor.connect(gain);
          gain.connect(audioContext.destination);

          scriptProcessor.onaudioprocess = (async (event: any) => {
            const bins = new Float32Array(analyserNode.frequencyBinCount);
            analyserNode.getFloatFrequencyData(bins);
            for (let i = 0; i < bins.length; i++) {
              analyserNodeData.push(bins[i]);
            }
            const audioFP = JSON.stringify(analyserNodeData);
            hash = CryptoJS.MD5(audioFP).toString();
            gain.disconnect();
            scriptProcessor.disconnect();
            analyserNode.disconnect();
            await audioContext.close();
            resolve({"hash": hash, "values": analyserNodeData, "noFingerprint": false});
          });
          // Load the Audio the first time through, otherwise play it from the buffer
          if(audioData == null) {
            const request = new XMLHttpRequest();
            request.open('GET', '../../../assets/viper-05.ogg', true);
            request.responseType = 'arraybuffer';
            request.onload = (() => {
              audioContext.decodeAudioData(request.response, function(buffer){
                audioData = buffer;
                sourceNode.buffer = buffer;
                sourceNode.start(0);    // Play the sound now
                sourceNode.loop = false;
              },
              function(e){"Error with decoding audio data" + e});
            });
            request.send()
          } else {
            sourceNode.buffer = audioData;
            sourceNode.start(0);    // Play the sound now
            sourceNode.loop = false;
          }
        } else {
          reject({"hash": hash, "values": analyserNodeData, "noFingerprint": true});
        }  
      } catch(u) {
        reject({"hash": hash, "values": analyserNodeData, "noFingerprint": true});
      }  
    });
  }
\end{lstlisting}

\begin{lstlisting}[language=JavaScript, caption=Channel Merge Hybrid fingerprint generation code, label=channelMergeHybridFingerprintGenerationCode]
  getChannelMergeHybridFingerprintAudioCtx(): Promise<any> {
    let hybridHash = null;
    let hybridOscillatorNode = [];
    return new Promise((resolve, reject) => {
      try {
        let audioCtx = new ((<any>window).AudioContext || (<any>window).webkitAudioContext)();
        if (audioCtx) {
          let oscillator1 = audioCtx.createOscillator();
          oscillator1.type = "sine";
          oscillator1.frequency.setValueAtTime(440, audioCtx.currentTime);
          let oscillator2 = audioCtx.createOscillator();
          oscillator2.type = "triangle";
          oscillator2.frequency.setValueAtTime(10000,audioCtx.currentTime);
          let oscillator3 = audioCtx.createOscillator();
          oscillator3.type = "square";
          oscillator3.frequency.setValueAtTime(1880,audioCtx.currentTime);
          let oscillator4 = audioCtx.createOscillator();
          oscillator4.type = "sawtooth";
          oscillator4.frequency.setValueAtTime(22000,audioCtx.currentTime);

          let channelMerger = audioCtx.createChannelMerger(4)
          oscillator1.connect(channelMerger, 0, 0);
          oscillator2.connect(channelMerger, 0, 1);
          oscillator3.connect(channelMerger, 0, 2);
          oscillator4.connect(channelMerger, 0, 3);

          let compressor = audioCtx.createDynamicsCompressor();
          let analyser = audioCtx.createAnalyser();
          let gain = audioCtx.createGain();
          let scriptProcessor = audioCtx.createScriptProcessor(SCRIPT_PROCESSOR.bufferSize, SCRIPT_PROCESSOR.numberOfInputChannels, SCRIPT_PROCESSOR.numberOfOutputChannels);

          compressor.threshold.setValueAtTime(COMPRESSOR.threshold, audioCtx.currentTime);
          compressor.knee.setValueAtTime(COMPRESSOR.knee, audioCtx.currentTime);
          compressor.ratio.setValueAtTime(COMPRESSOR.ratio, audioCtx.currentTime);
          compressor.attack.setValueAtTime(COMPRESSOR.attack, audioCtx.currentTime);
          compressor.release.setValueAtTime(COMPRESSOR.release, audioCtx.currentTime);

          gain.gain.value = 0; // Disable volume
          analyser.fftSize = 4096;
          
          channelMerger.connect(compressor); // Connect merger to compressor
          compressor.connect(analyser); // Connect compressor to analyser
          analyser.connect(scriptProcessor); // Connect analyser output to scriptProcessor input
          scriptProcessor.connect(gain); // Connect scriptProcessor output to gain input
          gain.connect(audioCtx.destination); // Connect gain output to audiocontext destination
    
          scriptProcessor.onaudioprocess = (async (bins: any) => {
              bins = new Float32Array(analyser.frequencyBinCount);
              analyser.getFloatFrequencyData(bins);
              for (let i = 0; i < bins.length; i++) {
                hybridOscillatorNode.push(bins[i]);
              }
              analyser.disconnect();
              scriptProcessor.disconnect();
              gain.disconnect();
              const audioFP = JSON.stringify(hybridOscillatorNode);
              hybridHash = CryptoJS.MD5(audioFP).toString();
              await audioCtx.close();
              resolve({"hash": hybridHash, "values": hybridOscillatorNode, "noFingerprint": false});
          });
          //start source
          oscillator1.start(0);
          oscillator2.start(0);
          oscillator3.start(0);
          oscillator4.start(0);
        } else {
          reject({"hash": hybridHash, "values": hybridOscillatorNode, "noFingerprint": true});
        }
      } catch (u) {
        reject({"hash": hybridHash, "values": hybridOscillatorNode, "noFingerprint": true});
      }
    });
  }
\end{lstlisting}

\begin{lstlisting}[language=JavaScript, caption=Amplitude Modulation Hybrid fingerprint generation code, label=amplitudeModualtionHybridFingerprintGenerationCode]
  getAmplitudeModulationHybridFingerprintAudioCtx(): Promise<any> {
    let hybridOscillatorNode = [];
    let hybridHash = null;
    return new Promise((resolve,reject) => {
      try {
        let audioCtx = new ((<any>window).AudioContext || (<any>window).webkitAudioContext)();
        if (audioCtx) {
          let mod = audioCtx.createOscillator();
          mod.frequency.setValueAtTime(18, audioCtx.currentTime);
          mod.type = "square"
  
          let modGain = audioCtx.createGain();
          modGain.gain.value = 30;
  
          let mod1 = audioCtx.createOscillator();
          mod1.frequency.setValueAtTime(440, audioCtx.currentTime);
          mod1.type = "triangle"
      
          let modGain1 = audioCtx.createGain();
          modGain1.gain.value = 60;
  
          let carrier = audioCtx.createOscillator();
          carrier.type = "sine"
          carrier.frequency.setValueAtTime(10000, audioCtx.currentTime);
          
          let carrierGain = audioCtx.createGain();
          carrierGain.gain.value = 1;

          let analyser = audioCtx.createAnalyser();
          let masterGain = audioCtx.createGain();
          masterGain.gain.value = 0; // Disable volume
          let scriptProcessor = audioCtx.createScriptProcessor(SCRIPT_PROCESSOR.bufferSize, SCRIPT_PROCESSOR.numberOfInputChannels, SCRIPT_PROCESSOR.numberOfOutputChannels);

          mod.connect(modGain);
          mod1.connect(modGain1);
          mod.connect(carrierGain.gain);
          mod1.connect(carrierGain.gain);
          carrier.connect(carrierGain);
          // Create and configure compressor
          let compressor = audioCtx.createDynamicsCompressor();
          compressor.threshold.setValueAtTime(COMPRESSOR.threshold, audioCtx.currentTime);
          compressor.knee.setValueAtTime(COMPRESSOR.knee, audioCtx.currentTime);
          compressor.ratio.setValueAtTime(COMPRESSOR.ratio, audioCtx.currentTime);
          compressor.attack.setValueAtTime(COMPRESSOR.attack, audioCtx.currentTime);
          compressor.release.setValueAtTime(COMPRESSOR.release, audioCtx.currentTime);
          
          analyser.fftSize = 4096;

          carrierGain.connect(compressor); // Connect carrier oscillator output to dynamic compressor
          compressor.connect(analyser); // Connect compressor to analyser
          analyser.connect(scriptProcessor); // Connect analyser output to scriptProcessor input
          scriptProcessor.connect(masterGain); // Connect scriptProcessor output to gain input
          masterGain.connect(audioCtx.destination); // Connect gain output to audiocontext destination
          scriptProcessor.onaudioprocess = (async (bins: any) => {
              bins = new Float32Array(analyser.frequencyBinCount);
              analyser.getFloatFrequencyData(bins);
              for (let i = 0; i < bins.length; i++) {
                  hybridOscillatorNode.push(bins[i]);
              }
              analyser.disconnect();
              scriptProcessor.disconnect();
              masterGain.disconnect();
              const audioFP = JSON.stringify(hybridOscillatorNode);
              hybridHash = CryptoJS.MD5(audioFP).toString();
              await audioCtx.close();
              resolve({"hash": hybridHash, "values": hybridOscillatorNode, "noFingerprint": false});
          });
          carrier.start(0);
          mod.start(0);
          mod1.start(0);
        } else {
          reject({"hash": hybridHash, "values": hybridOscillatorNode, "noFingerprint": true});
        }
      } catch (u) {
        reject({"hash": hybridHash, "values": hybridOscillatorNode, "noFingerprint": true});
      }
    });
  }
\end{lstlisting}

\begin{lstlisting}[language=JavaScript, caption=Frequency Modulation Hybrid  fingerprint generation code, label=frequencyModualtionHybridFingerprintGenerationCode]
  getFrequencyModulationHybridFingerprintAudioCtx(): Promise<any> {
    let hybridOscillatorNode = [];
    let hybridHash = null;
    return new Promise((resolve,reject) => {
      try {
        let audioCtx = new ((<any>window).AudioContext || (<any>window).webkitAudioContext)();
        if (audioCtx) {
          let mod = audioCtx.createOscillator();
          mod.frequency.setValueAtTime(18, audioCtx.currentTime);
          mod.type = "square"
  
          let modGain = audioCtx.createGain();
          modGain.gain.value = 30;
  
          let mod1 = audioCtx.createOscillator();
          mod1.frequency.setValueAtTime(440, audioCtx.currentTime);
          mod1.type = "triangle"
      
          let modGain1 = audioCtx.createGain();
          modGain1.gain.value = 60;
  
          let carrier = audioCtx.createOscillator();
          carrier.type = "sine"
          carrier.frequency.setValueAtTime(10000, audioCtx.currentTime);
          
          let carrierGain = audioCtx.createGain();
          carrierGain.gain.value = 1;

          let analyser = audioCtx.createAnalyser();
          let masterGain = audioCtx.createGain();
          masterGain.gain.value = 0; // Disable volume
          let scriptProcessor = audioCtx.createScriptProcessor(SCRIPT_PROCESSOR.bufferSize, SCRIPT_PROCESSOR.numberOfInputChannels, SCRIPT_PROCESSOR.numberOfOutputChannels);

          mod1.connect(modGain1);
          modGain1.connect(carrier.frequency);
    
          mod.connect(modGain);
          modGain.connect(carrier.frequency);

          // Create and configure compressor
          let compressor = audioCtx.createDynamicsCompressor();
          compressor.threshold.setValueAtTime(COMPRESSOR.threshold, audioCtx.currentTime);
          compressor.knee.setValueAtTime(COMPRESSOR.knee, audioCtx.currentTime);
          compressor.ratio.setValueAtTime(COMPRESSOR.ratio, audioCtx.currentTime);
          compressor.attack.setValueAtTime(COMPRESSOR.attack, audioCtx.currentTime);
          compressor.release.setValueAtTime(COMPRESSOR.release, audioCtx.currentTime);

          carrier.connect(compressor); // Connect carrier output to analyser input
          compressor.connect(analyser); // Connect compressor to analyser
          analyser.connect(scriptProcessor); // Connect analyser output to scriptProcessor input
          scriptProcessor.connect(masterGain); // Connect scriptProcessor output to gain input
          masterGain.connect(audioCtx.destination); // Connect gain output to audiocontext destination
          scriptProcessor.onaudioprocess = (async (bins: any) => {
              bins = new Float32Array(analyser.frequencyBinCount);
              analyser.getFloatFrequencyData(bins);
              for (let i = 0; i < bins.length; i++) {
                  hybridOscillatorNode.push(bins[i]);
              }
              analyser.disconnect();
              scriptProcessor.disconnect();
              masterGain.disconnect();
              const audioFP = JSON.stringify(hybridOscillatorNode);
              hybridHash = CryptoJS.MD5(audioFP).toString();
              await audioCtx.close();
              resolve({"hash": hybridHash, "values": hybridOscillatorNode, "noFingerprint": false});
          });
          carrier.start(0);
          mod.start(0);
          mod1.start(0);
        } else {
          reject({"hash": hybridHash, "values": hybridOscillatorNode, "noFingerprint": true});
        }
      } catch (u) {
        reject({"hash": hybridHash, "values": hybridOscillatorNode, "noFingerprint": true});
      }
    });
  }
\end{lstlisting}



\begin{lstlisting}[language=JavaScript, caption=Expanded Font list, label=expandedfontlist]
["sans-serif-thin","ARNO PRO","Agency FB","Arabic Typesetting","Arial Unicode MS","AvantGarde Bk BT","BankGothic Md BT","Batang","Bitstream Vera Sans Mono","Calibri","Century","Century Gothic","Clarendon","EUROSTILE","Franklin Gothic","Futura Bk BT","Futura Md BT","GOTHAM","Gill Sans","HELV","Haettenschweiler","Helvetica Neue","Humanst521 BT","Leelawadee","Letter Gothic","Levenim MT","Lucida Bright","Lucida Sans","Menlo","MS Mincho","MS Outlook","MS Reference Specialty","MS UI Gothic","MT Extra","MYRIAD PRO","Marlett","Meiryo UI","Microsoft Uighur","Minion Pro","Monotype Corsiva","PMingLiU","Pristina","SCRIPTINA","Segoe UI Light","Serifa","SimHei","Small Fonts","Staccato222 BT","TRAJAN PRO","Univers CE 55 Medium","Vrinda","ZWAdobeF",".Aqua Kana",".Helvetica LT MM",".Times LT MM","18thCentury","8514oem","AR BERKLEY","AR JULIAN","AR PL UKai CN","AR PL UMing CN","AR PL UMing HK","AR PL UMing TW","AR PL UMing TW MBE","Aakar","Abadi MT Condensed Extra Bold","Abadi MT Condensed Light","Abyssinica SIL","AcmeFont","Adobe Arabic","Agency FB","Aharoni","Aharoni Bold","Al Bayan","Al Bayan Bold","Al Bayan Plain","Al Nile","Al Tarikh","Aldhabi","Alfredo","Algerian","Alien Encounters","Almonte Snow","American Typewriter","American Typewriter Bold","American Typewriter Condensed","American Typewriter Light","Amethyst","Andale Mono","Andale Mono Version","Andalus","Angsana New","AngsanaUPC","Ani","AnjaliOldLipi","Aparajita","Apple Braille","Apple Braille Outline 6 Dot","Apple Braille Outline 8 Dot","Apple Braille Pinpoint 6 Dot","Apple Braille Pinpoint 8 Dot","Apple Chancery","Apple Color Emoji","Apple LiGothic Medium","Apple LiSung Light","Apple SD Gothic Neo","Apple SD Gothic Neo Regular","Apple SD GothicNeo ExtraBold","Apple Symbols","AppleGothic","AppleGothic Regular","AppleMyungjo","AppleMyungjo Regular","AquaKana","Arabic Transparent","Arabic Typesetting","Arial","Arial Baltic","Arial Black","Arial Bold","Arial Bold Italic","Arial CE","Arial CYR","Arial Greek","Arial Hebrew","Arial Hebrew Bold","Arial Italic","Arial Narrow","Arial Narrow Bold","Arial Narrow Bold Italic","Arial Narrow Italic","Arial Rounded Bold","Arial Rounded MT Bold","Arial TUR","Arial Unicode MS","ArialHB","Arimo","Asimov","Autumn","Avenir","Avenir Black","Avenir Book","Avenir Next","Avenir Next Bold","Avenir Next Condensed","Avenir Next Condensed Bold","Avenir Next Demi Bold","Avenir Next Heavy","Avenir Next Regular","Avenir Roman","Ayuthaya","BN Jinx","BN Machine","BOUTON International Symbols","BabyKruffy","Baghdad","Bahnschrift","Balthazar","Bangla MN","Bangla MN Bold","Bangla Sangam MN","Bangla Sangam MN Bold","Baskerville","Baskerville Bold","Baskerville Bold Italic","Baskerville Old Face","Baskerville SemiBold","Baskerville SemiBold Italic","Bastion","Batang","BatangChe","Bauhaus 93","Beirut","Bell MT","Bell MT Bold","Bell MT Italic","Bellerose","Berlin Sans FB","Berlin Sans FB Demi","Bernard MT Condensed","BiauKai","Big Caslon","Big Caslon Medium","Birch Std","Bitstream Charter","Bitstream Vera Sans","Blackadder ITC","Blackoak Std","Bobcat","Bodoni 72","Bodoni MT","Bodoni MT Black","Bodoni MT Poster Compressed","Bodoni Ornaments","BolsterBold","Book Antiqua","Book Antiqua Bold","Bookman Old Style","Bookman Old Style Bold","Bookshelf Symbol 7","Borealis","Bradley Hand","Bradley Hand ITC","Braggadocio","Brandish","Britannic Bold","Broadway","Browallia New","BrowalliaUPC","Brush Script","Brush Script MT","Brush Script MT Italic","Brush Script Std","Brussels","Calibri","Calibri Bold","Calibri Light","Californian FB","Calisto MT","Calisto MT Bold","Calligraphic","Calvin","Cambria","Cambria Bold","Cambria Math","Candara","Candara Bold","Candles","Carrois Gothic SC","Castellar","Centaur","Century","Century Gothic","Century Gothic Bold","Century Schoolbook","Century Schoolbook Bold","Century Schoolbook L","Chalkboard","Chalkboard Bold","Chalkboard SE","Chalkboard SE Bold","ChalkboardBold","Chalkduster","Chandas","Chaparral Pro","Chaparral Pro Light","Charlemagne Std","Charter","Chilanka","Chiller","Chinyen","Clarendon","Cochin","Cochin Bold","Colbert","Colonna MT","Comic Sans MS","Comic Sans MS Bold","Commons","Consolas","Consolas Bold","Constantia","Constantia Bold","Coolsville","Cooper Black","Cooper Std Black","Copperplate","Copperplate Bold","Copperplate Gothic Bold","Copperplate Light","Corbel","Corbel Bold","Cordia New","CordiaUPC","Corporate","Corsiva","Corsiva Hebrew","Corsiva Hebrew Bold","Courier","Courier 10 Pitch","Courier Bold","Courier New","Courier New Baltic","Courier New Bold","Courier New CE","Courier New Italic","Courier Oblique","Cracked Johnnie","Creepygirl","Curlz MT","Cursor","Cutive Mono","DFKai-SB","DIN Alternate","DIN Condensed","Damascus","Damascus Bold","Dancing Script","DaunPenh","David","Dayton","DecoType Naskh","Deja Vu","DejaVu LGC Sans","DejaVu Sans","DejaVu Sans Mono","DejaVu Serif","Deneane","Desdemona","Detente","Devanagari MT","Devanagari MT Bold","Devanagari Sangam MN","Didot","Didot Bold","Digifit","DilleniaUPC","Dingbats","Distant Galaxy","Diwan Kufi","Diwan Kufi Regular","Diwan Thuluth","Diwan Thuluth Regular","DokChampa","Dominican","Dotum","DotumChe","Droid Sans","Droid Sans Fallback","Droid Sans Mono","Dyuthi","Ebrima","Edwardian Script ITC","Elephant","Emmett","Engravers MT","Engravers MT Bold","Enliven","Eras Bold ITC","Estrangelo Edessa","Ethnocentric","EucrosiaUPC","Euphemia","Euphemia UCAS","Euphemia UCAS Bold","Eurostile","Eurostile Bold","Expressway Rg","FangSong","Farah","Farisi","Felix Titling","Fingerpop","Fixedsys","Flubber","Footlight MT Light","Forte","FrankRuehl","Frankfurter Venetian TT","Franklin Gothic Book","Franklin Gothic Book Italic","Franklin Gothic Medium","Franklin Gothic Medium Cond","Franklin Gothic Medium Italic","FreeMono","FreeSans","FreeSerif","FreesiaUPC","Freestyle Script","French Script MT","Futura","Futura Condensed ExtraBold","Futura Medium","GB18030 Bitmap","Gabriola","Gadugi","Garamond","Garamond Bold","Gargi","Garuda","Gautami","Gazzarelli","Geeza Pro","Geeza Pro Bold","Geneva","GenevaCY","Gentium","Gentium Basic","Gentium Book Basic","GentiumAlt","Georgia","Georgia Bold","Geotype TT","Giddyup Std","Gigi","Gill","Gill Sans","Gill Sans Bold","Gill Sans MT","Gill Sans MT Bold","Gill Sans MT Condensed","Gill Sans MT Ext Condensed Bold","Gill Sans MT Italic","Gill Sans Ultra Bold","Gill Sans Ultra Bold Condensed","Gisha","Glockenspiel","Gloucester MT Extra Condensed","Good Times","Goudy","Goudy Old Style","Goudy Old Style Bold","Goudy Stout","Greek Diner Inline TT","Gubbi","Gujarati MT","Gujarati MT Bold","Gujarati Sangam MN","Gujarati Sangam MN Bold","Gulim","GulimChe","GungSeo Regular","Gungseouche","Gungsuh","GungsuhChe","Gurmukhi","Gurmukhi MN","Gurmukhi MN Bold","Gurmukhi MT","Gurmukhi Sangam MN","Gurmukhi Sangam MN Bold","Haettenschweiler","Hand Me Down S (BRK)","Hansen","Harlow Solid Italic","Harrington","Harvest","HarvestItal","Haxton Logos TT","HeadLineA Regular","HeadlineA","Heavy Heap","Hei","Hei Regular","Heiti SC","Heiti SC Light","Heiti SC Medium","Heiti TC","Heiti TC Light","Heiti TC Medium","Helvetica","Helvetica Bold","Helvetica CY Bold","Helvetica CY Plain","Helvetica LT Std","Helvetica Light","Helvetica Neue","Helvetica Neue Bold","Helvetica Neue Medium","Helvetica Oblique","HelveticaCY","HelveticaNeueLT Com 107 XBlkCn","Herculanum","High Tower Text","Highboot","Hiragino Kaku Gothic Pro W3","Hiragino Kaku Gothic Pro W6","Hiragino Kaku Gothic ProN W3","Hiragino Kaku Gothic ProN W6","Hiragino Kaku Gothic Std W8","Hiragino Kaku Gothic StdN W8","Hiragino Maru Gothic Pro W4","Hiragino Maru Gothic ProN W4","Hiragino Mincho Pro W3","Hiragino Mincho Pro W6","Hiragino Mincho ProN W3","Hiragino Mincho ProN W6","Hiragino Sans GB W3","Hiragino Sans GB W6","Hiragino Sans W0","Hiragino Sans W1","Hiragino Sans W2","Hiragino Sans W3","Hiragino Sans W4","Hiragino Sans W5","Hiragino Sans W6","Hiragino Sans W7","Hiragino Sans W8","Hiragino Sans W9","Hobo Std","Hoefler Text","Hoefler Text Black","Hoefler Text Ornaments","Hollywood Hills","Hombre","Huxley Titling","ITC Stone Serif","ITF Devanagari","ITF Devanagari Marathi","ITF Devanagari Medium","Impact","Imprint MT Shadow","InaiMathi","Induction","Informal Roman","Ink Free","IrisUPC","Iskoola Pota","Italianate","Jamrul","JasmineUPC","JavaneseText","Jokerman","JuiceITC","KacstArt","KacstBook","KacstDecorative","KacstDigital","KacstFarsi","KacstLetter","KacstNaskh","KacstOffice","KacstOne","KacstPen","KacstPoster","KacstQurn","KacstScreen","KacstTitle","KacstTitleL","Kai","Kai Regular","KaiTi","Kailasa","Kailasa Regular","Kaiti SC","Kaiti SC Black","Kalapi","Kalimati","Kalinga","Kannada MN","Kannada MN Bold","Kannada Sangam MN","Kannada Sangam MN Bold","Kartika","Karumbi","Kedage","Kefa","Kefa Bold","Keraleeyam","Keyboard","Khmer MN","Khmer MN Bold","Khmer OS","Khmer OS System","Khmer Sangam MN","Khmer UI","Kinnari","Kino MT","KodchiangUPC","Kohinoor Bangla","Kohinoor Devanagari","Kohinoor Telugu","Kokila","Kokonor","Kokonor Regular","Kozuka Gothic Pr6N B","Kristen ITC","Krungthep","KufiStandardGK","KufiStandardGK Regular","Kunstler Script","Laksaman","Lao MN","Lao Sangam MN","Lao UI","LastResort","Latha","Leelawadee","Letter Gothic Std","LetterOMatic!","Levenim MT","LiHei Pro","LiSong Pro","Liberation Mono","Liberation Sans","Liberation Sans Narrow","Liberation Serif","Likhan","LilyUPC","Limousine","Lithos Pro Regular","LittleLordFontleroy","Lohit Assamese","Lohit Bengali","Lohit Devanagari","Lohit Gujarati","Lohit Gurmukhi","Lohit Hindi","Lohit Kannada","Lohit Malayalam","Lohit Odia","Lohit Punjabi","Lohit Tamil","Lohit Tamil Classical","Lohit Telugu","Loma","Lucida Blackletter","Lucida Bright","Lucida Bright Demibold","Lucida Bright Demibold Italic","Lucida Bright Italic","Lucida Calligraphy","Lucida Calligraphy Italic","Lucida Console","Lucida Fax","Lucida Fax Demibold","Lucida Fax Regular","Lucida Grande","Lucida Grande Bold","Lucida Handwriting","Lucida Handwriting Italic","Lucida Sans","Lucida Sans Demibold Italic","Lucida Sans Typewriter","Lucida Sans Typewriter Bold","Lucida Sans Unicode","Luminari","Luxi Mono","MS Gothic","MS Mincho","MS Outlook","MS PGothic","MS PMincho","MS Reference Sans Serif","MS Reference Specialty","MS Sans Serif","MS Serif","MS UI Gothic","MT Extra","MV Boli","Mael","Magneto","Maiandra GD","Malayalam MN","Malayalam MN Bold","Malayalam Sangam MN","Malayalam Sangam MN Bold","Malgun Gothic","Mallige","Mangal","Manorly","Marion","Marion Bold","Marker Felt","Marker Felt Thin","Marlett","Martina","Matura MT Script Capitals","Meera","Meiryo","Meiryo Bold","Meiryo UI","MelodBold","Menlo","Menlo Bold","Mesquite Std","Microsoft","Microsoft Himalaya","Microsoft JhengHei","Microsoft JhengHei UI","Microsoft New Tai Lue","Microsoft PhagsPa","Microsoft Sans Serif","Microsoft Tai Le","Microsoft Tai Le Bold","Microsoft Uighur","Microsoft YaHei","Microsoft YaHei UI","Microsoft Yi Baiti","Minerva","MingLiU","MingLiU-ExtB","MingLiU_HKSCS","Minion Pro","Miriam","Mishafi","Mishafi Gold","Mistral","Modern","Modern No. 20","Monaco","Mongolian Baiti","Monospace","Monotype Corsiva","Monotype Sorts","MoolBoran","Moonbeam","MotoyaLMaru","Mshtakan","Mshtakan Bold","Mukti Narrow","Muna","Myanmar MN","Myanmar MN Bold","Myanmar Sangam MN","Myanmar Text","Mycalc","Myriad Arabic","Myriad Hebrew","Myriad Pro","NISC18030","NSimSun","Nadeem","Nadeem Regular","Nakula","Nanum Barun Gothic","Nanum Gothic","Nanum Myeongjo","NanumBarunGothic","NanumGothic","NanumGothic Bold","NanumGothicCoding","NanumMyeongjo","NanumMyeongjo Bold","Narkisim","Nasalization","Navilu","Neon Lights","New Peninim MT","New Peninim MT Bold","News Gothic MT","News Gothic MT Bold","Niagara Engraved","Niagara Solid","Nimbus Mono L","Nimbus Roman No9 L","Nimbus Sans L","Nimbus Sans L Condensed","Nina","Nirmala UI","Nirmala.ttf","Norasi","Noteworthy","Noteworthy Bold","Noto Color Emoji","Noto Emoji","Noto Mono","Noto Naskh Arabic","Noto Nastaliq Urdu","Noto Sans","Noto Sans Armenian","Noto Sans Bengali","Noto Sans CJK","Noto Sans Canadian Aboriginal","Noto Sans Cherokee","Noto Sans Devanagari","Noto Sans Ethiopic","Noto Sans Georgian","Noto Sans Gujarati","Noto Sans Gurmukhi","Noto Sans Hebrew","Noto Sans JP","Noto Sans KR","Noto Sans Kannada","Noto Sans Khmer","Noto Sans Lao","Noto Sans Malayalam","Noto Sans Myanmar","Noto Sans Oriya","Noto Sans SC","Noto Sans Sinhala","Noto Sans Symbols","Noto Sans TC","Noto Sans Tamil","Noto Sans Telugu","Noto Sans Thai","Noto Sans Yi","Noto Serif","Notram","November","Nueva Std","Nueva Std Cond","Nyala","OCR A Extended","OCR A Std","Old English Text MT","OldeEnglish","Onyx","OpenSymbol","OpineHeavy","Optima","Optima Bold","Optima Regular","Orator Std","Oriya MN","Oriya MN Bold","Oriya Sangam MN","Oriya Sangam MN Bold","Osaka","Osaka-Mono","OsakaMono","PCMyungjo Regular","PCmyoungjo","PMingLiU","PMingLiU-ExtB","PR Celtic Narrow","PT Mono","PT Sans","PT Sans Bold","PT Sans Caption Bold","PT Sans Narrow Bold","PT Serif","Padauk","Padauk Book","Padmaa","Pagul","Palace Script MT","Palatino","Palatino Bold","Palatino Linotype","Palatino Linotype Bold","Papyrus","Papyrus Condensed","Parchment","Parry Hotter","PenultimateLight","Perpetua","Perpetua Bold","Perpetua Titling MT","Perpetua Titling MT Bold","Phetsarath OT","Phosphate","Phosphate Inline","Phosphate Solid","PhrasticMedium","PilGi Regular","Pilgiche","PingFang HK","PingFang SC","PingFang TC","Pirate","Plantagenet Cherokee","Playbill","Poor Richard","Poplar Std","Pothana2000","Prestige Elite Std","Pristina","Purisa","QuiverItal","Raanana","Raanana Bold","Raavi","Rachana","Rage Italic","RaghuMalayalam","Ravie","Rekha","Roboto","Rockwell","Rockwell Bold","Rockwell Condensed","Rockwell Extra Bold","Rockwell Italic","Rod","Roland","Rondalo","Rosewood Std Regular","RowdyHeavy","Russel Write TT","SF Movie Poster","STFangsong","STHeiti","STIXGeneral","STIXGeneral-Bold","STIXGeneral-Regular","STIXIntegralsD","STIXIntegralsD-Bold","STIXIntegralsSm","STIXIntegralsSm-Bold","STIXIntegralsUp","STIXIntegralsUp-Bold","STIXIntegralsUp-Regular","STIXIntegralsUpD","STIXIntegralsUpD-Bold","STIXIntegralsUpD-Regular","STIXIntegralsUpSm","STIXIntegralsUpSm-Bold","STIXNonUnicode","STIXNonUnicode-Bold","STIXSizeFiveSym","STIXSizeFiveSym-Regular","STIXSizeFourSym","STIXSizeFourSym-Bold","STIXSizeOneSym","STIXSizeOneSym-Bold","STIXSizeThreeSym","STIXSizeThreeSym-Bold","STIXSizeTwoSym","STIXSizeTwoSym-Bold","STIXVariants","STIXVariants-Bold","STKaiti","STSong","STXihei","SWGamekeys MT","Saab","Sahadeva","Sakkal Majalla","Salina","Samanata","Samyak Devanagari","Samyak Gujarati","Samyak Malayalam","Samyak Tamil","Sana","Sana Regular","Sans","Sarai","Sathu","Savoye LET Plain:1.0","Sawasdee","Script","Script MT Bold","Segoe MDL2 Assets","Segoe Print","Segoe Pseudo","Segoe Script","Segoe UI","Segoe UI Emoji","Segoe UI Historic","Segoe UI Semilight","Segoe UI Symbol","Serif","Shonar Bangla","Showcard Gothic","Shree Devanagari 714","Shruti","SignPainter-HouseScript","Silom","SimHei","SimSun","SimSun-ExtB","Simplified Arabic","Simplified Arabic Fixed","Sinhala MN","Sinhala MN Bold","Sinhala Sangam MN","Sinhala Sangam MN Bold","Sitka","Skia","Skia Regular","Skinny","Small Fonts","Snap ITC","Snell Roundhand","Snowdrift","Songti SC","Songti SC Black","Songti TC","Source Code Pro","Splash","Standard Symbols L","Stencil","Stencil Std","Stephen","Sukhumvit Set","Suruma","Sylfaen","Symbol","Symbole","System","System Font","TAMu_Kadambri","TAMu_Kalyani","TAMu_Maduram","TSCu_Comic","TSCu_Paranar","TSCu_Times","Tahoma","Tahoma Negreta","TakaoExGothic","TakaoExMincho","TakaoGothic","TakaoMincho","TakaoPGothic","TakaoPMincho","Tamil MN","Tamil MN Bold","Tamil Sangam MN","Tamil Sangam MN Bold","Tarzan","Tekton Pro","Tekton Pro Cond","Tekton Pro Ext","Telugu MN","Telugu MN Bold","Telugu Sangam MN","Telugu Sangam MN Bold","Tempus Sans ITC","Terminal","Terminator Two","Thonburi","Thonburi Bold","Tibetan Machine Uni","Times","Times Bold","Times New Roman","Times New Roman Baltic","Times New Roman Bold","Times New Roman Italic","Times Roman","Tlwg Mono","Tlwg Typewriter","Tlwg Typist","Tlwg Typo","TlwgMono","TlwgTypewriter","Toledo","Traditional Arabic","Trajan Pro","Trattatello","Trebuchet MS","Trebuchet MS Bold","Tunga","Tw Cen MT","Tw Cen MT Bold","Tw Cen MT Italic","URW Bookman L","URW Chancery L","URW Gothic L","URW Palladio L","Ubuntu","Ubuntu Condensed","Ubuntu Mono","Ukai","Ume Gothic","Ume Mincho","Ume P Gothic","Ume P Mincho","Ume UI Gothic","Uming","Umpush","UnBatang","UnDinaru","UnDotum","UnGraphic","UnGungseo","UnPilgi","Untitled1","Urdu Typesetting","Uroob","Utkal","Utopia","Utsaah","Valken","Vani","Vemana2000","Verdana","Verdana Bold","Vijaya","Viner Hand ITC","Vivaldi","Vivian","Vladimir Script","Vrinda","Waree","Waseem","Waverly","Webdings","WenQuanYi Bitmap Song","WenQuanYi Micro Hei","WenQuanYi Micro Hei Mono","WenQuanYi Zen Hei","Whimsy TT","Wide Latin","Wingdings","Wingdings 2","Wingdings 3","Woodcut","X-Files","Year supply of fairy cakes","Yu Gothic","Yu Mincho","Yuppy SC","Yuppy SC Regular","Yuppy TC","Yuppy TC Regular","Zapf Dingbats","Zapfino","Zawgyi-One","gargi","lklug","mry_KacstQurn","ori1Uni"]
\end{lstlisting}
%%%%%%%%%%%%%%%%%%%%%%%%%%%%%%%%%%%%%%%%%%%%%%%%%%%%%%%%%%%%%%%%%%%%%%
%%                           APPENDIX B
%%%%%%%%%%%%%%%%%%%%%%%%%%%%%%%%%%%%%%%%%%%%%%%%%%%%%%%%%%%%%%%%%%%%%

\chapter{Figures}
\begin{figure}[H]
    \centering
    \subfloat[\centering Dynamic Compressor]{{\includegraphics[scale=0.5]{CDF/DynamicCompressor_CDF.jpeg}}}%
    \qquad
    \subfloat[\centering Oscillator Node]{{\includegraphics[scale=0.5]{CDF/OscillatorNode_CDF.jpeg}}}%
    \qquad
    \subfloat[\centering Hybrid]{{\includegraphics[scale=0.5]{CDF/Hybrid_CDF.jpeg}}}%
    \qquad
    \subfloat[\centering Custom Signal Hybrid]{{\includegraphics[scale=0.5]{CDF/CustomSignalHybrid_CDF.jpeg}}}%
    \qquad
    \caption[{The frequency of the highest occurring fingerprint value in 30 iterations for individual audio fingerprinting methods contd.}]{The frequency of the highest occurring fingerprint value in 30 iterations for individual audio fingerprinting methods contd.}
    \label{fig:CDFshowingAllFpsContd.}%
\end{figure}
\begin{figure}[H]
    \centering
    \ContinuedFloat
     \subfloat[\centering Audio Source Hybrid]{{\includegraphics[scale=0.5]{CDF/AudioSourceHybrid_CDF.jpeg}}}%
    \qquad
    \subfloat[\centering Channel Merge Hybrid]{{\includegraphics[scale=0.5]{CDF/ChannelMergeHybrid_CDF.jpeg}}}%
    \qquad
    \subfloat[\centering Amplitude Modulation Hybrid]{{\includegraphics[scale=0.5]{CDF/AmplitudeModulationHybrid_CDF.jpeg}}}%
    \qquad
    \subfloat[\centering Frequency Modulation Hybrid]{{\includegraphics[scale=0.5]{CDF/FrequencyModulationHybrid_CDF.jpeg}}}%
    \caption[{The frequency of the highest occurring fingerprint value in 30 iterations for individual audio fingerprinting methods}]{The frequency of the highest occurring fingerprint value in 30 iterations for individual audio fingerprinting methods}
    \label{fig:CDFshowingAllFps}%
\end{figure}

\pagebreak{}

\end{appendices}

\phantomsection
% \addcontentsline{toc}{chapter}{Vita}
% \phantomsection

\chapter*{Vita}
The author, Saroj Duwal, was born in Bhaktapur, Nepal. He obtained his Bachelor's degree in Computer Science with a minor in Mathematics from the University of New Orleans in 2019. He joined the University of New Orleans Computer Science graduate program to pursue a Masters in Computer Science in Spring 2020. He has been working under Dr. Ben Samuels within the department of Computer Science at the University of New Orleans.


\end{document}
