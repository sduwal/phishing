
%%%%%%%%%%%%%%%%%%%%%%%%%%%%%%%%%%%%%%%%%%%%%%%%%%%
%%%%%%%%%%%%%%%%%%%%%%%%%%%%%%%%%%%%%%%%%%%%%%%%%%%%%%%%%%%%%%%%%%%%%
%%                           ABSTRACT
%%%%%%%%%%%%%%%%%%%%%%%%%%%%%%%%%%%%%%%%%%%%%%%%%%%%%%%%%%%%%%%%%%%%%

\chapter*{Abstract}
\addcontentsline{toc}{chapter}{Abstract} % Needs to be set to part, so the TOC doesn't add 'CHAPTER ' prefix in the TOC.

\pagestyle{plain} % No headers, just page numbers
\pagenumbering{roman} % Roman numerals
\setcounter{page}{8}
Phishing attacks are challenging to detect and can have severe consequences. In 2020 alone, phishing attacks cost organizations more than \$1.8 billion. In addition, attacks can have effects other than money, as shown by the infamous case of John Podesta during the 2016 US presidential election, in which staffs were tricked into sharing passwords by fake Google security emails, granting access to confidential information. Vulnerabilities such as these are partly due to insufficient and tiresome user training. We can minimize such vulnerabilities with better and more engaging training against phishing. To address this, we have designed and developed an interactive game to teach users phishing concepts by placing the player as an attacker. Our user study shows that our game was engaging, and after playing the game, participants had a better understanding of phishing and recognizing phishing emails.

\vspace{1em}
\noindent
\textbf{Keywords}: Anti-Phishing, Serious games, Cyber Security, Training
\pagebreak{}
