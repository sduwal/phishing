
%%%%%%%%%%%%%%%%%%%%%%%%%%%%%%%%%%%%%%%%%%%%%%%%%%%
%%%%%%%%%%%%%%%%%%%%%%%%%%%%%%%%%%%%%%%%%%%%%%%%%%%%%%%%%%%%%%%%%%%%%
%%                           ABSTRACT 
%%%%%%%%%%%%%%%%%%%%%%%%%%%%%%%%%%%%%%%%%%%%%%%%%%%%%%%%%%%%%%%%%%%%%

\chapter*{Abstract}
\addcontentsline{toc}{chapter}{Abstract} % Needs to be set to part, so the TOC doesn't add 'CHAPTER ' prefix in the TOC.

\pagestyle{plain} % No headers, just page numbers
\pagenumbering{roman} % Roman numerals
\setcounter{page}{8}
% Browser fingerprinting presents a grave threat to the privacy of internet users as it allows user tracking even in private browsing modes. Prior measurement studies on HTML5-based fingerprinting have been limited to Canvas and WebGL APIs but not Web Audio APIs. We aim to fill this gap by conducting \emph{the first large-scale systematic study of web audio fingerprints} and studying their~\emph{stability} as well as~\emph{diversity} properties. Using MTurk and other social media platforms, we collected audio fingerprints from 633 web users. As part of this, we designed and implemented 5 new audio fingerprint vectors by obtaining FFTs of audio source files and modulated custom waveforms.

% Our results present several interesting insights into the nature of Web Audio fingerprints. Firstly, we show that the audio fingerprints are unstable unlike other fingerprinting methods with some users having as many as 20 different fingerprints. Despite this, we show that audio fingerprinting can still be used as an effective fingerprinting vector as the most popular fingerprints do tend to get repeated quite often. In order to do this, we devised a graph-based fingerprint matching mechanism and used it to measure the diversity of audio fingerprints. Our results show that audio fingerprints are much less diverse than other vectors with only 39 distinct fingerprints among 633 users. However, further analysis showed that contrary to what is currently believed by W3C, these fingerprints have additional fingerprinting value in comparison to the~\texttt{User-Agent} HTTP headers. Overall, our results allow browser developers to gauge the degree of privacy invasion presented by audio fingerprinting  thus helping them take a more informed stance when designing privacy protection features.

Browser fingerprinting presents a grave threat to privacy as it allows user tracking even in private browsing modes. Prior measurement studies on HTML5-based fingerprinting have been limited to Canvas and WebGL but not Web Audio APIs. We aim to fill this gap by conducting \emph{the first large-scale systematic study of web audio fingerprints} and studying their~\emph{stability} as well as~\emph{diversity} properties. Using MTurk and social media platforms, we collected 8 different audio fingerprints from 694 users.

Firstly, we show that the audio fingerprints are unstable unlike other fingerprinting methods with some users having as many as 20 different fingerprints. Despite this, we show that audio fingerprinting can still be used as an effective fingerprinting vector as most fingerprints tend to repeat quite often. We devised a graph-based fingerprint matching mechanism to measure the diversity of audio fingerprints. Our results show that audio fingerprints are much less diverse with only 45 distinct fingerprints among 694 users.



\textbf{Keywords}: Web Audio Fingerprints, Browser Fingerprints, Web Audio API, Canvas Fingerprints, Font Fingerprints, User-Agents Fingerprints, Tracking, Web Security
\pagebreak{}
