%%%%%%%%%%%%%%%%%%%%%%%%%%%%%%%%%%%%%%%%%%%%%%%%%%%%%%%%%%%%%%%%%%%%%%
%%                           INTRODUCTION
%%%%%%%%%%%%%%%%%%%%%%%%%%%%%%%%%%%%%%%%%%%%%%%%%%%%%%%%%%%%%%%%%%%%%


\pagestyle{plain} % No headers, just page numbers
\pagenumbering{arabic} % Arabic numerals
\setcounter{page}{1}
\chapter{Introduction}

The rapid adoption of the internet and emails in the workspace and everyday life has presented new security challenges. Phishing attacks are one of the most common and well-known problems experienced by many organizations \cite{jampen}.  Phishing attacks are reported to be on the rise and unlikely to decrease soon. A report by Anti-Phishing Working Group (APWG) saw the number of phishing attacks has doubled from early 2020 \cite{APWG}. A common type of phishing attack is a \emph{click-jacking attack} that is designed to trick users into clicking on links that are not intended for them. Such links redirect victims to sites that are carefully designed to mimic those of legitimate businesses and services with the goal of convincing users to provide their personal information and credentials\cite{jampen}. Effects of such attacks are hard to quantify all attacks may not be reported. Estimates of damage caused by phishing vary widely, ranging from \$ 61 million per year to \$ 3 billion per year \cite{hong_2012}.

Phishing attacks are often carried out as a starting point for other detrimental cyber-attacks\cite{jampen}. Attackers ofter start by targeting the weakest link i.e. the human layer\cite{accenture}. There are various technical security measures in place to prevent and filter out phishing emails\cite{hird, dou}, but such systems are not perfect and cannot filter out all malicious emails.

Training the human layer is a critical step in preventing and filtering out phishing emails. There have been plenty of research on the best way to train the human layer. One of such prevalent way of training the users are with the use of cyber security videos and tutorials. However, studies show that such resources work only if the users are paying attention to them\cite{kumaraguru_sheng_acquisti_cranor_hong_2010}. To better engage the learner, several anti-phishing games have been proposed. Although, study shows that games have been successful in improving the users ability to detect spam links and websites, many games leave out the email context. Attackers might leverage these to demand immediate attention and encourage quick response. Attackers might use different techniques such as spoofing and present sense of urgency in the email to trick the users in clicking the email.

To develop an comprehensive training game, we have developed "game", a role playing simulation game. Role playing game to train users are not a novel concept\cite{what_hack}, and have shown promising result. However, existing role playing games approach the training from the victim perspective. This might limit the emails they see and miss important details on what to look for. "Game" requires the player to act as an attacker and train different skills used to create phishing emails. Phishing emails are randomly generated (based on the attacker skills) from a set of templates collected from real phishing emails. Role playing as an attacker also allows us to concentrate on domains, and implement some mechanics to present how different domains and subdomains can be used to create phishing emails.

